\documentclass{article}
\usepackage[utf8]{inputenc}
\usepackage[english,russian]{babel}
\usepackage{amsmath, amsfonts, amssymb, amsthm, mathtools }
\usepackage{ upgreek }

\title{Домашнее задание №2}
\author{Амир Капкаев}
\date{March 2020}

\usepackage{natbib}
\usepackage{graphicx}

\begin{document}

\maketitle
\newpage
(И) Вычисление частного и остатка от деления числа, заданного в унарной системе
счисления, на два (над алфавитом $\sum = \{1, \#\}$). Результат должен записываться
в виде «частное\#остаток». Ноль должен соответствовать пустому слову. \\
$\sum' = \sum \cup \{*\}$\\
\\
$\left\{
\begin{aligned}
*11 &\rightarrow 1*\\
*1 &\rightarrow .\#1\\
* &\rightarrow .\#\\
11 &\rightarrow *11\\
1 &\rightarrow \#1 \\
\end{aligned}
\right.$\\
\\
(К)  Дублирование всех символов входного слова (над алфавитом $\sum = \{a, b\}$).\\
$\sum' = \sum \cup \{*\}$  \\
\\
$\left\{
\begin{aligned}
*a &\rightarrow aa* \\
*b &\rightarrow bb* \\
\epsilon &\rightarrow * \\
* &\rightarrow .\epsilon 
\end{aligned}
\right.$\\
\\
(Л) Перестановка символов входного слова в обратном порядке (над $\sum = \{a, b\}$).\\
$\sum' = \sum \cup \{*,\#,|\}$ \\
\\
$\left\{
\begin{aligned}
*a &\rightarrow a*\\
*b &\rightarrow b*\\
* &\rightarrow \#\\
aa\# &\rightarrow a\#a\\
ba\# &\rightarrow a\#b\\
ab\# &\rightarrow b\#a\\
bb\# &\rightarrow b\#b\\
|a\# &\rightarrow a|*\\
|b\# &\rightarrow b|*\\
|* &\rightarrow .\epsilon\\
a &\rightarrow |*a\\
b &\rightarrow |*b
\end{aligned}
\right.$ \\
\\
\newpage
(М) Cортировка символов входного слова (над алфавитом $\sum = \{a, b, c\}$).\\
\\
$\left\{
\begin{aligned}
ba &\rightarrow ab \\
cb &\rightarrow bc \\
ca &\rightarrow ac \\ 
\end{aligned}
\right.$ 
\\
(Н) Проверка, является ли входное слово палиндромом (над алфавитом $\sum = \{a, b\}$).
Если является, то результатом должно быть пустое слово, если не является, то
результатом может быть любое непустое слово.
$\sum' = \sum \cup \{*, \#\}$\\
$\left\{
\begin{aligned}
a*a &\rightarrow aa*\\
a*b &\rightarrow ba*\\
b*a &\rightarrow ab*\\
b*b &\rightarrow bb*\\
* &\rightarrow \#\\
aa\# &\rightarrow \epsilon\\
bb\# &\rightarrow \epsilon\\
a\# &\rightarrow .\epsilon\\
b\# &\rightarrow .\epsilon\\
a &\rightarrow a*\\
b &\rightarrow b*\\
\end{aligned}
\right.$ \\
\\
(О) Проверка, является ли входное слово именем одного из основных регистров
процессора Intel 8088 (AX, BX, CX или DX). Результатом должно быть либо
имя регистра, либо пустое слово.\\
$\sum = \{A, B, C, ..., X, Y, Z\}$ \\
$\sum' = \sum \cup \{*, \#, <, >\}$\\
$\left\{
\begin{aligned}
E &\rightarrow >\\
F &\rightarrow >\\
&...\\
Z &\rightarrow >\\
>A &\rightarrow >\\
>B &\rightarrow >\\
...\\
>Z &\rightarrow >\\
> &\rightarrow \epsilon\\
A< &\rightarrow <\\
B< &\rightarrow <\\
&...\\
Z< &\rightarrow <\\
\#A &\rightarrow A\#\\
\#B &\rightarrow B\#\\
&...\\
\#Z &\rightarrow Z\#\\
AX\# &\rightarrow .AX\\
BX\# &\rightarrow .BX\\
CX\# &\rightarrow .CX\\
DX\# &\rightarrow .DX\\
\# &\rightarrow <\\
< &\rightarrow \epsilon\\
*X &\rightarrow X\#\\
*A &\rightarrow <>\\
*B &\rightarrow <>\\
&...\\
*Z &\rightarrow <>\\
A &\rightarrow A*\\
B &\rightarrow B*\\
C &\rightarrow C*\\
D &\rightarrow D*\\
\end{aligned}
\right.$ \\
\end{document}
